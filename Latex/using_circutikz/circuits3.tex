\documentclass{standalone}
\usepackage{siunitx}
\usepackage{circuitikz}
\begin{document}
\begin{circuitikz}
	\ctikzset{logic ports/scale=0.4}
\ctikzset{logic ports=ieee}
	\draw[step=1cm, gray!50, very thin] (-0.9,-0.9) grid(13,13);
	\foreach \x in {0,...,13}
	{
		\draw[black!50] (\x cm, 1pt) -- (\x cm, -1pt) 
		node[anchor=north]{$\x$};
	}
	\foreach \y in {0,...,13}
	{
		\draw[black!50] (1pt,\y cm) -- (-1pt,\y cm) 
		node[left]{$\y$};
	}


	\ctikzset{multipoles/thickness=4}
	\ctikzset{multipoles/external pins thickness=2}
	\draw (5,5) node[dipchip,
					num pins=14,
					hide numbers,
					external pins width=0.3,
					external pad fraction=4 ](C){};
	\draw (4.7,6) node[and port,rotate=-90,](and1){};
	\draw (4.7,4.4) node[and port,rotate=-90,](and2){};
	\draw (5.3,3.8) node[and port,rotate=-90,](and3){3};
	\draw (5.3,5.5) node[and port,rotate=-90,](and4){};

	\draw (C.bpin 1) -| (and1.in 1);
	\draw (C.bpin 2) -- ([xshift=0.2cm]C.bpin 2) |- (and1.in 2);
	\draw (C.bpin 3) -| (and1.out);

	\draw (C.bpin 4) -| (and2.in 1);
	\draw (C.bpin 5) -- ([xshift=0.2cm]C.bpin 5) |- (and2.in 2);
	\draw (C.bpin 6) -| (and2.out);

	\draw (C.bpin 10) -| (and3.in 2);
	\draw (C.bpin 9) -- ([xshift=-0.2cm]C.bpin 9) |- (and3.in 1);
	\draw (C.bpin 8) -| (and3.out);

	\draw (C.bpin 13) -| (and4.in 2);
	\draw (C.bpin 12) -- ([xshift=-0.2cm]C.bpin 12) |- (and4.in 1);
	\draw (C.bpin 11) -| (and4.out);

	\foreach \pinright in {1,...,7}
	{
		\draw node[right,font=\tiny] at(C.pin \pinright) {$\pinright$};
	}
	\foreach \pinleft in {8,...,14}
	{
		\draw node[left,font=\tiny] at (C.pin \pinleft) {$\pinleft$};
	}
	\node[font=\tiny] at (5,5){\text{\si{\raiseto{2}{R}}}};
\end{circuitikz}
\end{document}
